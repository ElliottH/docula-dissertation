\documentclass{beamer}
\usecolortheme[accent=blue]{solarized}
\beamertemplatetransparentcovered

\usepackage{lmodern}
\usepackage[UKenglish]{datetime}

\usepackage{listings}
\lstset{
    upquote=true,
    columns=fixed,
    tabsize=4,
    extendedchars=true,
    breaklines=true,
%    numbers=left,
    numbersep=5pt,
    backgroundcolor=\color{solarizedBase2},
    rulesepcolor=\color{solarizedBase03},
    numberstyle=\tiny\color{solarizedBase01},
    basicstyle=\footnotesize\ttfamily,
    keywordstyle=\color{solarizedGreen},
    stringstyle=\color{solarizedCyan}\ttfamily,
    identifierstyle=\color{solarizedBlue},
    commentstyle=\color{solarizedBase01},
    emphstyle=\color{solarizedRed}
}

\title{A Documentation Generation Engine}
\subtitle{Progress Report}

\author{Elliott Hillary}
\date{\today}

\begin{document}
  \begin{frame}
    \titlepage
  \end{frame}
  \begin{frame}
    \frametitle{Project Aims}
      \begin{itemize}
        \item Documentation is hard to get right, particularly if it isn't
          written during development
          \begin{itemize}
            \item An effective documentation process needs to be as convenient
              as possible for the developer
          \end{itemize}
          \item Software already exists to tackle this problem (e.g.~doxygen),
            but I aimed to approach it differently
            \begin{itemize}
              \item Interface documentation
                \begin{itemize}
                  \item \emph{What} it does
                \end{itemize}
              \item Implementation documentation
                  \begin{itemize}
                    \item \emph{How \& why} it does it
                  \end{itemize}
            \end{itemize}
          \item Modular by design
          \item Compatibility
      \end{itemize}
  \end{frame}

  \begin{frame}[fragile]
    \frametitle{Sample Docstring}
    \begin{lstlisting}[language=c, gobble=5, escapechar=~]
      /**
       * The realloc() function tries to change the size of
       * the allocation pointed to by ptr to size, and
       * returns ptr.
       * ...
       * ~{\color{solarizedCyan} @param[in, out]}~ ptr Pointer to the memory.
       * ~{\color{solarizedCyan} @param[in]}~ size (New) size of memory desired.
       * ~{\color{solarizedCyan} @return}~ If successful, returns a pointer to
       *        allocated memory. If there is an error,
       *        returns a NULL pointer and sets errno to
       *        ENOMEM.
       */
      void *realloc(void *ptr, unsigned int size);
    \end{lstlisting}
\end{frame}

  \begin{frame}
    \frametitle{Parsing is \emph{Hard}}
    \begin{itemize}
      \item Initial parser was {\raise.17ex\hbox{$\scriptstyle\sim$}}500 LOC and
        successfully parsed about 95\% of C
      \item At this point adding the final few things was time-consuming at
        best, and seemingly impossible at worst
        \begin{itemize}
          \item In the interests of advancing on schedule, this was infeasible
        \end{itemize}
      \item ...And so a simplified parser was born
        \begin{itemize}
          \item Weighs in at half LOC of its predecessor
        \end{itemize}
      \item Simplification has its drawbacks
        \begin{itemize}
          \item Function bodies go largely unparsed, making referencing harder
        \end{itemize}
    \end{itemize}
  \end{frame}

  \begin{frame}
    \frametitle{Current State}
    \begin{itemize}
      \item Simplified parser
        \begin{itemize}
          \item Knows enough C to pull out what we need and safely navigate the
            rest
          \item Stumbles on macros \& similar odd things people do in header
            files
        \end{itemize}
      \item One input language, one output format
        \begin{itemize}
          \item C $\rightarrow$ HTML
        \end{itemize}
      \item Basic command line utility
        \begin{itemize}
          \item Traverses directory hierarchy looking for things it cares about,
            with a few options
        \end{itemize}
    \end{itemize}
  \end{frame}

  \begin{frame}
    \frametitle{Next Steps}
      \begin{itemize}
        \item Begin dissertation work
        \item Cross-referencing
          \begin{itemize}
            \item Types and functions defined in different compilation units
              should explicitly refer to one another
          \end{itemize}
        \item Parser optimisation \& improvements
          \begin{itemize}
            \item Parser currently trips over certain things, and should
              probably do so more gracefully (and hopefully not at all!)
          \end{itemize}
        \item Track references to functions (and variables?)
          \begin{itemize}
            \item Maintenance of a large source tree is much easier if you know
              what's calling what without digging
          \end{itemize}
      \end{itemize}
  \end{frame}
\end{document}
