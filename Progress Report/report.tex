\documentclass[10pt,oneside,a4paper]{article}
\usepackage{setspace}
\onehalfspacing
\begin{document}

\vfil

\centerline{\Large Progress Report}
\vspace{0.4in}
\centerline{\Large A Documentation Generation Engine }
\vspace{0.4in}
\centerline{\large E. Hillary, Girton College}
\vspace{0.3in}
\centerline{\large ejh67@cam.ac.uk}
\vspace{0.3in}
\centerline{\large 30$^{th}$ January 2013}

\vfil


\noindent
{\bf Project Supervisor:} Dr.~R.~R.~Watts
\vspace{0.2in}

\noindent
{\bf Director of Studies:} C.~K.~Hadley
\vspace{0.2in}
\noindent

\noindent
{\bf Project Overseers:} Prof.~R.~J.~Anderson  \& Prof.~J.~M.~Bacon

\vfil
\newpage

\section*{Progress Made}
Currently, the project is capable of taking valid C code and producing a basic
HTML output of documentation extracted from the code; at the time of writing,
the HTML output is relatively un-styled \& basic, however it is within the scope
of the project to make the output functional rather than aesthetically pleasing.
Improvements to the design of the output were originally to be improved in weeks
3--5 of this term, and may well still be depending on the ease of resolution of
other problems.

At the moment, very little cross-referencing occurs between compilation units,
so, for example, a type defined in a header file and referred to in a source
file will not link back to its definition. This has not been completed due to
time-constraints and problems encountered in other parts of the project,
mentioned below.

According to the original timetable put forward in my Project Proposal, the two
weeks at the start of Lent term were allocated for initial testing on real-world
projects; while some has taken place, it has produced poorer results than
anticipated. This will mean that more time will be needed to resolve these
problems than I expected, and as such optimisations and extensions to the
project are unlikely to take place.

\section*{Problems Encountered}
Creating a Parsing Expression Grammar for C proved much harder than anticipated,
and as a result writing the parser took about two weeks longer than expected.
Implementation was troublesome due to the breadth \& depth of the C
specification, and the parser got to the point where its complexity was so great
that making progress in one area broke another.

This problem was resolved by creating a new, simpler parser, that understood
enough C to know how to parse out the required parts. This included things like
not parsing function bodies any more than was necessary to work out where the
end of the function was. Fortunately, writing the initial parser wasn't a waste
of time, as it made me much more familiar with the oddities of C syntax and with
PEG parsers and allowed the writing of the simpler parser to take place quickly.
It also allowed me to learn first-hand how to deal with left recursion problems
in parsing.
\\
\\
I also found it difficult to implement my project in the order I had originally
laid out in the timetable, where I had listed that the output stage would be
written before the processing stage; I had originally put them in this order
because I thought it would help increase the modularity of the project. However,
this proved troublesome because it was hard to visualise what structure the data
would be in for the output stage to work with without knowing what was feasible
to do in the processing stage. Unfortunately, the result of this is that there
isn't a clearly distinct processing stage in the project any more, rather the
output stage works directly from the data stored by the input stage.
\\
\\
Given the problems encountered, Dr.~Watts and I plan to discuss if any
replanning is required; however a meeting between us was unable to take place
in time to add the appropriate information to this progress report, our
discussion will have taken place by the progress meeting.
\end{document}
