\section{Speed Testing}
\begin{center}
\begin{tabular}{l r || c | c | c}
  Source   & Software & Second Run & Actual Time & Projected Time \\
  \hline
  DirectFB & docula   & 36.583     & 289.111     & 322.320        \\
           & doxygen  &            & 17.410      &                \\
  \hline
  freetype &          & 1.738      & 31.723      & 43.394         \\
           &          &            & 4.621       &                \\
  \hline
  ffmpeg   &          & 36.629     & 213.540     & 250.371        \\
           &          &            & 50.366      &                \\
  \hline
  icu      &          & 7.491      & 83.869      & 100.566        \\
           &          &            & 58.007      &                \\
  \hline
  VTK      &          & 68.028     & 391.183     & 502.976        \\
           &          &            & 362.26      &                \\
\end{tabular}
\end{center}

\includepdf[pages={1},angle=90]{Graphs/timings.pdf}

\section{Correctness Testing}
\begin{longtable}{l || c | c | c}
  \multicolumn{1}{c||}{File} & \multicolumn{1}{c|}{Displayed} & \multicolumn{1}{c|}{Grouping} & \multicolumn{1}{c}{Docstrings} \\
  \hline
  \endhead

  \hline \multicolumn{4}{r}{{Continued on next page}} \\
  \endfoot

  \endlastfoot

  aes.h                 & 3         & 2        & 5          \\
  atomic.h              & 4         & 0        & 4          \\
  attributes.h          & 1         & 0        & 3          \\
  audio\_fifo.c         & 1         & 0        & 8          \\
  audio\_fifo.h         & 11        & 2        & 13          \\
  avassert.h            & 2         & 0        & 4          \\
  base64.c              & 1         & 0        & 2          \\
  base64.h              & 3         & 2        & 5          \\
  blowfish.h            & 3         & 2        & 5          \\
  buffer\_internal.h    & 2         & 0        & 8          \\
  channel\_layout.h     & 14        & 5        & 17          \\
  colorspace.h          & 1         & 0        & 1          \\
  des.h                 & 3         & 0        & 3          \\
  error.h               & 5         & 2        & 7          \\
  file.h                & 4         & 0        & 4          \\
  imgutils.c            & 1         & 0        & 1          \\
  imgutils.h            & 12        & 2        & 13          \\
  integer.c             & 1         & 0        & 1          \\
  intfloat\_readwrite.c & 1         & 0        & 1          \\
  lfg.h                 & 3         & 0        & 3          \\
  lls.c                 & 1         & 0        & 1          \\
  lls.h                 & 1         & 0        & 1          \\
  lzo.c                 & 6         & 0        & 4          \\
  lzo.h                 & 6         & 4        & 5          \\
  md5.h                 & 0         & 2        & 2          \\
  mem.c                 & 1         & 0        & 1          \\
  opencl.h              & 19        & 0        & 19          \\
  parseutils.h          & 9         & 0        & 9          \\
  pca.c                 & 1         & 0        & 1          \\
  pca.h                 & 1         & 0        & 1          \\
  pixdesc.h             & 15        & 0        & 20          \\
  ppc/cpu.c             & 1         & 0        & 1          \\
  ppc/types\_altivec.h  & 2         & 0        & 2          \\
  qsort.h               & 2         & 0        & 2          \\
  random\_seed.h        & 1         & 2        & 3          \\
  rational.c            & 1         & 0        & 1          \\
  rc4.h                 & 2         & 0        & 2          \\
  sha.c                 & 1         & 0        & 2          \\
  sha.h                 & 4         & 2        & 6          \\
  time.h                & 2         & 0        & 2          \\
  timer.h               & 1         & 0        & 1          \\
  timestamp.h           & 5         & 0        & 5          \\
  version.h             & 4         & 7        & 8          \\
  xga\_font\_data.c     & 1         & 0        & 1          \\
  xtea.h                & 2         & 2        & 4          \\
\end{longtable}

\includepdf[pages={1},angle=90]{Graphs/correct.pdf}

\section{Functionality Testing}
\emph{The questionnaire used to survey users can be found later in the
Appendix}

\begin{center}
\begin{tabular}{l || c | c | c | c | c | c | c | c | c | c | c | c | c | c}
   & Q1 & Q2 & Q3 & Q4 & Q5 & Q6 & Q7 & Q8 & Q9 & Q10 & Q11 & Q12 & Q13 & Q14 \\
\hline
\#1 & Y  & 3  & 7  & 6  & Y  & 2  & Y  & 3  & 4  & Y   & Y   & 5   & 4   & 3  \\
\#2 & Y  & 4  & 7  & 5  & Y  & 3  & N  & 2  & 3  & Y   & Y   & 2   & 4   & 3  \\
\#3 & N  &    &    & 7  & N  & 3  & Y  & 3  & 4  & N   & N   & 3   & 4   & 2  \\
\#4 & Y  & 3  & 8  &    & Y  & 1  & N  &    & 2  & Y   & N   & 5   & 3   & 3  \\
\#5 & Y  & 3  & 6  & 6  & Y  & 2  & Y  & 3  & 2  & Y   & N   & 4   & 3   & 3  \\
\#6 & Y  & 2  & 8  & 8  & Y  & 3  & Y  & 5  & 5  & Y   & N   & 5   & 5   & 4  \\
\end{tabular}
\end{center}

\begin{landscape}
\begin{center}
\begin{longtable}{l || p{3cm} | p{7cm} | p{8cm}}
  \multicolumn{1}{c||}{} & \multicolumn{1}{c|}{Current S/W} & \multicolumn{1}{c|}{Missing Features} & \multicolumn{1}{c}{Additional Comments} \\
  \hline
  \endhead

  \hline \multicolumn{4}{r}{{Continued on next page}} \\
  \endfoot

  \endlastfoot
  \#1
    & Doxygen, Sandcastle, Javadoc
    & Perhaps some search functionality?
    & Maybe include frames to allow easy navigation within the hierarchy at any
      scroll point in the document. \\
\hline
  \#2
    & restructuredtext / docutils
    &
    & Saying ``Failed to parse rational.h with CSimpleParser!'' makes it sound
      like a SERIOUS ERROR (keying on the word ``Failed'' and the exclamation
      mark). Leaving off the exclamation mark and replacing Failed with (for
      instance) ``Unable'' might be better. Or maybe list what *was* parsed
      instead, and ignore those not...

      It feels worrying that there are lots of Failed messages - there's no
      scope to give one a feeling if this is a problem or not. \\
\hline
  \#3
    & The files on the top-level page don't appear to be ordered in an obvious
      manner?
    & It would be useful to have function names emphasized in the "Function
      Documentation" section.
    & Docula failed to recognise function arguments split over several lines and
      @param tags. Otherwise the operation was fine. \\
\hline
  \#4
    & doxygen
    & Any sort of navigation between files.
    & My parameter documentation didn't appear after editing time.h, the rest
      did. \\
\hline
  \#5
    & doxygen
    & Finding functions was a matter of guesswork, there was no automated help.
      This is probably an unsolveable problem if your front-end is a browser.
      No links back to files.html.
      Several files didn't produce anything, and docula gave no clue why. It did
      spam the output a lot with "Failed to parse xxx.x with CSimpleParser!"
      which is uninformative to a user and not configence-inspiring.

      files.html doesn't list files in alphabetical order. Or any particularly
      obvious order for that matter, even directory hierarchies are only
      reflected in the filename. This makes finding things harder.
    & docula's own documentation is ironically lacking. There is no man page,
      which is fair enough, and if gems come with documentation I have no idea
      how to find it. More seriously, giving docula an invalid "-?" option on
      the command line does not produce a syntax message, just a ruby traceback.

      Personally I'd like a "vicious bastard" mode that errors messily if the
      documentation is incomplete.

      It feels like docula is following the same path as doxygen of being
      focused on functions, but offering very limited help with structure. My
      big problems are all structure-related, the lack of documentation about
      how things fit together, and docula doesn't seem to provide any support
      for that. \\
\hline
  \#6
    & doxygen
    & Being able to search through the documentation, other than just the
      current page.
    & When I added comments to time.h, I couldn't get it to generate any text
      about the argument to the sleep function.

      I like the different background shadings used to differentiate entries in
      a section. \\
\end{longtable}
\end{center}
\end{landscape}
