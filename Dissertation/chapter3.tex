% This chapter should describe what was actually produced: the programs which
% were written, the hardware which was built or the theory which was developed.
% Any design strategies that looked ahead to the testing stage might profitably
% be referred to (the professional approach again).

% Descriptions of programs may include fragments of high-level code but large
% chunks of code are usually best left to appendices or omitted altogether.
% Analogous advice applies to circuit diagrams.

% Draw attention to the parts of the work which are not your own. Making
% effective use of powerful tools and pre-existing code is often laudable, and
% will count to your credit if properly reported.

% It should not be necessary to give a day-by-day account of the progress of the
% work but major milestones may sometimes be highlighted with advantage.

\chapter{Implementation}

\section{The Parser}
To implement my parser, I made use of an existing domain-specific language
available as a Ruby library called Treetop\cite{website:treetop}; Treetop allows
for easy implementation of Parsing Expression Grammars, which, simply put, are
grammars that built up out of a combination of rules and regular expressions.

Parsing Expression Grammars were particularly appealing to me for two reasons:
firstly, they are essentially a more powerful version of regular expressions,
with which I am already familiar and secondly, they can be made to run in
constant time if implemented as a packrat parser, which Treetop does. This
constant time guarantee is provided by the process of memoization, which trades
off time for space by caching previous inputs and the resulting parse trees to
remove the need for repeated computations.

Treetop grammars are written as their own .treetop files, which are then used
to generate the parsers by the \lstinline|tt| program. The end result is a Ruby
source file which can be included and used to parse the language defined by the
grammar. Treetop also allows you to define a class that nodes within the parsed
data should be instantiated as, this means that nodes can be given methods with
which they can parse themselves or their children; which is much preferable to
the traditional methods of writing code to walk the tree structures.

\begin{code}[language=treetop]
  rule comment
    '//' ( !EOL .)* EOL <CommentNode>
    / '# ' ( !EOL .)* EOL <CommentNode>
    / (multiline_comment) <CommentNode>
  end
  rule multiline_comment
    '/*'
    (
      !'*/'
      (. / EOL)
    )*
    '*/'
  end
  rule EOL
    [\n]
  end
\end{code}

Above is a snippet from my Treetop grammar for C, defining the grammar for
comments. The rule \lstinline|comment| contains an ordered choice of expressions
that define each type of comment available in C, and informs Treetop that they
should be instantiated using the class \lstinline|CommentNode|;
\lstinline|CommentNode| defines a method with which the text of the comment
can be extracted.


  \subsection{Regression Testing}
    To ensure that I was properly implementing my grammar for C, I wrote some
    tests for my parser that would check that parsing a particular piece of code
    would yield a valid result. This was the quickest \& easiest way of building
    up a working grammar for the language and allowed me to be confident that
    what I had written would function correctly as I moved forward.

    This was implemented using \lstinline|TestCase|, a class that is part of the
    Ruby standard library, and simply involves defining methods that execute the
    code required for each test and uses various types of assertions to ensure
    that the results are as expected.

    The process went as follows:

    \begin{center}
    \begin{tikzpicture}[node distance = 2cm, auto]
        % Place nodes
        \node [block] (init) {write tests for new syntax};
        \node [block, below=of init] (run) {run the tests};
        \node [decision, below=of run] (fail) {does it fail?};
        \node [block, left=of fail] (debug) {another rule in the grammar is
          parsing incorrectly};
        \node [block, below=of fail, node distance=3cm] (develop) {add new rules
          to the grammar};
        \node [block, below=of develop, node distance=3cm] (rerun) {re-run the
          test};
        \node [decision, right=of rerun] (refail) {does it fail?};
        % Draw edges
        \path [line] (init) -- (run);
        \path [line] (run) -- (fail);
        \path [line] (fail) -- node[decision answer] {no} (debug);
        \path [line] (fail) -- node[decision answer] {yes} (develop);
        \path [line] (develop) -- (rerun);
        \path [line] (rerun) -- (refail);
        \path [line] (refail.east) |- node[decision answer] {no} (init);
        \path [line] (refail) |- node[decision answer] {yes} (develop);
    \end{tikzpicture}
    \end{center}

  \subsection{Parsing Problems}
    As I wrote my initial grammar for the C parser, it became apparent that I
    was having some problems; I got to the point where implementing new rules
    within the grammar would cause other tests to fail, the problem was that
    describing C correctly required such a huge grammar that the rules were
    interacting in ways I had not accounted for. This meant that progress on the
    grammar slowed to a crawl, and it became apparent that the grammar would
    need to be largely rethought to fully parse C.

    I realised, however, that in order to parse C for the purposes of generating
    documentation, I did not need my parser to completely understand the
    language, instead it needed to only understand enough to parse to top level.
    I therefore wrote a new, simpler parser that understood how to navigate
    through the C files, rather than parsing all of it; this brought an added
    advantage of improving the runtime of the parser.

    Instead of parsing function bodies, the simplified parser contains rules to
    match pairs of braces within functions, ignoring those in strings and
    comments, to understand where functions begin \& end. With global variable
    assignments, the parser has rules to match from an equals sign to the
    semicolon at the end of the line. These simplifications allowed for me to
    dispense with the expensive and complicated parsing of control structures
    and arithmetic, and reduced the grammar from ${\sim}$500 lines in the
    original (unfinished) grammar to ${\sim}$300 lines for the completed
    simplified one.

    Ideally, some or all of the more complex behaviour would be restored to the
    grammar, as it would allow for better analysis of the parsed language, such
    as including in the documentation what was referenced by a particular
    function. Given the time constraints involved in the project, this could not
    be performed.

\section{Handling the Data}
After the source files have been parsed into a parse tree, the useful data can
be extracted; using the methods that were added into the parse tree by Treetop,
the code runs through the functions, variables \& other pieces of data and pulls
out the needed information to be stored in the database.

To store the data I opted to use SQLite, as it requires no configuration to use
and stores all data in a file, rather than on a SQL server; this means that no
setup of databases is required on the part of the user. I used the
sqlite3\cite{website:sqlite} Ruby library to interact with my database, which
allows you to interact with data from the database using the normal Ruby types.

  \subsection{Processing}
    The classes I defined for use in the parse tree provide easy access to the
    data within the nodes, and their subtrees, whilst masking the specifics of
    the tree itself; this means that future additions to the parser and the way
    in which it works will not affect the processing of the data. In addition,
    this allows for the reuse of the processing code on the parse trees of
    different languages, provided that the nodes in that tree expose the same
    methods.

    Since some functions in the source being parsed may only be partially
    documented, I needed to account for this in my code to ensure that the
    resulting data was consistent; in particular, if a function is documented
    the hash containing information about the prototype will contain extra
    information. Uniting the arguments with their documentation proved
    interesting, and I had to write the code for managing it several times
    before I created a solution that provided an appropriate output regardless
    of input.

    \begin{code}[language=ruby, gobble=6]
      if function.documented?
        prototype[:arguments] = prototype[:arguments].zip(prototype[:params]).map do |a,d|
          d ? a.merge(d) : a
        end
        prototype.delete(:params)
      end
    \end{code}

    This small section of code was the eventual solution, and in just a few
    lines accomplishes quite a bit. The array containing the arguments
    parsed, from the function prototype itself, is merged with the array
    containing the documentation for each argument, producing a new array. This
    array is contains several arrays, each holding the documentation and
    prototype for one argument; finally, the map function returns the array I
    need by merging the two hashes together, if the argument is documented, or
    returning the just the argument's hash if no documentation exists. The
    example data below demonstrates the process.

    \begin{code}[language=ruby, gobble=6]
      A = [{a => 1}, {b => 2}, {c => 3}]
      B = [{d => 4}, {}, {f => 6}]

      # After zipping A & B...

      C = [[{a => 1}, {d => 4}], [{b => 2}], [{c => 3}, {f => 6}]]

      # After mapping C...

      D = [{a => 1, d => 4}, {b => 2}, {c => 3, f => 6}]
    \end{code}

      \subsubsection{Post-processing}
        Once the data for all files in the directory has been added to the
        database, some extra processing takes place to resolve which files
        include one another. By looping through all the includes encountered
        and determining whether it refers another file that has been processed,
        links are built between files, which are later used by the output stage
        to allow a user to jump from one file to one of its includes easily.

        To determine if one file refers to another, I decided to check whether a
        following the path described in the include exists, relative to that
        file. This is by no means a fool-proof solution, as it does not take
        into account the include paths passed to a compiler, but the only other
        sensible solution to this problem would be to also parse the Makefile
        (or similar file in another build system) to extract the include paths
        used. Given the number of different build systems used and the time
        taken to implement another grammar for parsing any of them, I concluded
        that this was not a valid solution in the time available.

  \subsection{Storage}


    \subsubsection{Foreign Keys}

\section{Generating the Output}

Once all the data has been processed and stored, the output can be generated;
currently there is only one available output format, HTML, however the
outputting process has been designed so that a new output format can be easily
added. To do this, a generic \lstinline|Output| class was created that defines
methods that can be used to extract all the data required for producing the
output. This means that creation of a new output format only requires the
definition of a new subclass of \lstinline|Output| that defines an
\lstinline|output| method, which can call the other methods already defined to
get its data.

Generating output can be a messy process, especially with HTML, as inevitably
code and HTML markup end up being interleaved together; I have done my best to
separate the two out by defining functions that produce particular sections of
commonly repeated HTML, such as creating links and table rows, to increase the
readability of the code, however there is only so much this can be done.

\begin{code}[language=ruby, gobble=2]
  def row_with_id(id, *entries)
    if entries == []
      "<tr id=\"#{id}\"></tr>"
    else
      "<tr id=\"#{id}\"><td>#{entries.join("</td><td>")}</td></tr>\n"
    end
  end
\end{code}

The function above allows me to create a new row in a table
\lstinline|row_with_id("some_id", ["item1", "item2"])| when I need one, rather
than repeating the same HTML over \& over. String interpolation also helps the
situation greatly, as it allowed me to write much more concise code; the above
code without it would have been much more verbose and complicated, line 5 would
have been:

\begin{code}[language=ruby, gobble=2]
  "<tr id=\"" << id.to_s << "\"><td>" << entries.join("</td><td>") << "</td></tr>\n"
\end{code}

This would have also been more expensive to compute, as it requires mutating the
string four times, rather than just creating it correctly in the first place.
This is one example of a reason I chose Ruby in the in my proposal.

  \subsection{Linking Between Files}
    Linking between the outputted pages, for things like the list of includes,
    was solved using \lstinline|Pathname|, another class available in the Ruby
    standard library. \lstinline|Pathname| allows you to determine the relative
    path between two files; since I had ensured that the directory structure of
    my outputted data matched that of the input data, the relative path between
    the two source files is also the relative path between the HTML files. This
    meant that producing these paths was fairly easy.

    Taking this a step further, I also implemented linking a type to the file it
    was originally declared in; this was done with \lstinline|Pathname| as
    above, and a \lstinline|type| method in \lstinline|Output| that looks up
    types in the SQLite database.

    \begin{code}[language=sql, gobble=6]
      SELECT types.*, files.path FROM
       (SELECT id, file_id, name, 'typedef' FROM typedefs WHERE typedefs.name = ?
        UNION ALL
        SELECT id, file_id, name, 'define' FROM defines WHERE defines.name = ?
       ) AS 'types' INNER JOIN files ON types.file_id = files.id
    \end{code}

    This SQL statement looks up a type in the \lstinline|typedef| \&
    \lstinline|define| tables, unioning the two result sets together to ensure
    that we get results from either, and then performs an inner join of these
    results against the files table to get the information on the file that the
    typedef or define was found in. This results in a rather large SQL
    statement, which can be a little complicated to understand, however it is
    much quicker than running three queries and producing the result in Ruby.

\section{The Command-Line Interface}
The command line interface ties all of the previous stages together, allowing
for everything to be used together conveniently without requiring any Ruby
knowledge on the part of the user.

Aside from making the process more convenient, the command line interface also
provides easy manipulation of the options the program uses and performs the
cleaning and the initial setup of the database.

Within this same section of code is also where the traversal of directories to
find the files to be parsed takes place; Ruby's use of blocks allows the code
for this to be written quickly and clearly. I wrote a small method,
\lstinline|recurse|, that recurses down through the directories contained within
a particular path, and yields the files contained in them back to its caller.
This way of doing this made working through the files as convenient as working
through the elements of an array, without having to store all of the filenames
in an array first.

% \begin{lstlisting}[language=ruby, gobble=2]
%   # Compare:

%   ["test.h", "test.c", "another_file.txt"].each do |element|
%     if parsers[File.extname(element)]
%       # ...
%     end
%   end

%   # versus

%   Dir.recurse(options[:directory]) do |file|
%     if parsers[File.extname(file)]
%       # ...
%     end
%   end
% \end{lstlisting}

The process goes as follows:

\begin{center}
    \begin{tikzpicture}[node distance = 0.5cm, auto]
        % Place nodes
        \node [block] (exist) {mark all database entries as exist=0};
        \node [block, left=of exist] (check) {check extension};
        \node [decision, below=of check] (extension) {correct?};
        \node [decision, left=of extension] (next) {next file};
        \node [block, below=of extension] (md5) {generate MD5 of file};
        \node [decision, below=of md5] (match) {hash match?};
        \node [block, below=of next] (mark) {mark file as exist=1};
        \node [block, below=of match] (parse) {parse the file};
        \node [decision, below=of parse] (success) {success?};
        \node [block, below=of success] (process) {process the parse tree};
        \node [block, left=of next] (delete) {delete database entries where
          exist=0};
        % Draw edges
        \path [line] (exist) -- (check);
        \path [line] (check) -- (extension);
        \path [line] (extension) -- node[decision answer] {yes} (md5);
        \path [line] (extension) -- node[decision answer] {no} (next);
        \path [line] (md5) -- (match);
        \path [line] (match) -| node[decision answer] {yes} (mark);
        \path [line] (match) -- node[decision answer] {no} (parse);
        \path [line] (parse) -- (success);
        \path [line] (success) -| node[decision answer] {no} (mark);
        \path [line] (success) -- node[decision answer] {yes} (process);
        \path [line] (mark) -- (next);
        \path [line] (process) -| (mark);
        \path [line] (next) |- node[decision answer] {yes} (check);
        \path [line] (next) -- node[decision answer] {no} (delete);
    \end{tikzpicture}
    \end{center}

  \subsection{Option Parsing}
    Ruby has an extremely robust command-line option parser built into the
    standard library, so there was no need to roll my own or make use of an
    external library. \lstinline|OptionParser| made defining options simple, and
    produces help text automatically, so the whole process was quite straight
    forward.

    The following snippet is sufficient to describe a verbose option for a
    program:

    \begin{lstlisting}[language=ruby,gobble=6]
      options = {}
      OptionParser.new do |opts|
        opts.banner = "Usage: example.rb [options]"

        opts.on("-v", "--[no-]verbose", "Run verbosely") do |v|
          options[:verbose] = v
        end
      end.parse!
    \end{lstlisting}

    Using \lstinline|OptionParser| allowed me to specify default options within
    my \lstinline|options| hash, which can then be overridden by the user on the
    command-line.

    \lstinline|OptionParser| also allows you to define an option that takes only
    certain inputs, using this it was easy to create and option for choosing the
    output format; even though there is only one currently, this project was
    designed with extensibility in mind.

    \begin{code}[language=ruby, gobble=6]
      o.on("--format FORMAT", outputs, "Select output format",
           "  (#{outputs.keys.join(", ")})") do |format|
        options[:format] = format
      end
    \end{code}

    This adds an option that will accept any entry that is a key in the
    \lstinline|outputs| hash, and sets the appropriate entry in the
    \lstinline|options| hash to the value of the entry, which is the output
    class to use. This means that adding a new output format requires only a
    subclass of \lstinline|Output| and an entry referencing it in the
    \lstinline|outputs| hash.
