\chapter{Comment Format}

\definecolor{solarized@base03}{HTML}{002B36}
\definecolor{solarized@base02}{HTML}{073642}
\definecolor{solarized@base01}{HTML}{586e75}
\definecolor{solarized@base00}{HTML}{657b83}
\definecolor{solarized@base0}{HTML}{839496}
\definecolor{solarized@base1}{HTML}{93a1a1}
\definecolor{solarized@base2}{HTML}{EEE8D5}
\definecolor{solarized@base3}{HTML}{FDF6E3}
\definecolor{solarized@yellow}{HTML}{B58900}
\definecolor{solarized@orange}{HTML}{CB4B16}
\definecolor{solarized@red}{HTML}{DC322F}
\definecolor{solarized@magenta}{HTML}{D33682}
\definecolor{solarized@violet}{HTML}{6C71C4}
\definecolor{solarized@blue}{HTML}{268BD2}
\definecolor{solarized@cyan}{HTML}{2AA198}
\definecolor{solarized@green}{HTML}{859900}

\lstset{
    upquote=true,
    columns=fixed,
    tabsize=4,
    extendedchars=true,
    breaklines=true,
%    numbers=left,
    numbersep=5pt,
%    backgroundcolor=\color{solarized@base2},
    rulesepcolor=\color{solarized@base03},
    numberstyle=\tiny\color{solarized@base01},
    basicstyle=\footnotesize\ttfamily,
    keywordstyle=\color{solarized@green},
    stringstyle=\color{solarized@cyan}\ttfamily,
    identifierstyle=\color{solarized@blue},
    commentstyle=\color{solarized@base01},
    emphstyle=\color{solarized@red}
}

\begin{lstlisting}[language=c]
    /**
     * In order to ensure compatibility with Doxygen, I needed to create a
     * comment format that was as similar to theirs as possible.
     * Any block comment with a double star opening will be considered for a
     * documentation string.
     */

     /// Triple slashes opening a single line comment are also considered.

     /**
      * Documentation immediately preceding a function definition will become
      * associated with that function.
      *
      * Typically, the first paragraph is a summary, with the subsequent
      * paragraphs providing greater detail; this can be disabled in the
      * options.
      *
      * @param[in] Description of first argument
      * @param[inout] Description of second argument
      * @return Description of return value
      * @error Describes what make be returned in the case of an error.
      *
      * The @param annotations are for describing the arguments passed into
      * the function - what they represent and/or the range or valid values -
      * they also specify whether the argument is used solely for input,
      * output, or both ([in], [out] and [inout], respectively).
      */
    int example(int arg1, char* arg2);

    /// Documentation may also precede variables...
    int important_var;

    /// ... Or even #defines.
    #define ULTIMATE_ANSWER 42

    /**
     * @file
     *
     * Documentation marked with the @file annotation denote that they are
     * referring to the file as a whole.
     *
     * These comments may also include other relevant information, such as:
     * @copyright My Company Inc.
     * @author Elliott Hillary
     * @created 2012-10-18
     */
\end{lstlisting}
