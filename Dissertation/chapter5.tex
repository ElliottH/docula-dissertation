% This chapter is likely to be very short and it may well refer back to the
% Introduction. It might properly explain how you would have planned the project
% if starting again with the benefit of hindsight.

\chapter{Conclusion}
I have successfully produced software capable of transforming source code, using
appropriately-styled comments, into human-readable documentation. The software
does not completely parse the C language correctly, mainly due to difficulties
in parsing preprocessor macros; given more time, this should be fairly easy to
fix. As the results in the previous chapter show, the software parses source
trees faster on subsequent runs than doxygen, due to the use of persistent
storage.

Whilst the software I have produced does indeed produce documentation, it does
not address the shortcomings that I identified in existing documentation
generation software, namely that they do not allow the developer to fully
describe ``the design decisions that led to this implementation nor the
considerations that drove them''. Due in part to time constraints, difficulties
and hold-ups in the implementation process, the software I have produced only
goes as far as existing software does in displaying documentation, however I
believe that I have designed my system in a modular enough way that these things
could be added in the future without any significant re-designs taking place.

I originally opted to use Treetop for creating the parser because it allowed me
to implement it in a way that was logical and familiar to me, with hindsight, it
would perhaps have been better (at least, in the time that was available to me)
to make use of an infrastructure such as LLVM\cite{website:llvm} to facilitate
the parsing. This would have also given me the added advantage of being able to
perform some static analysis on the code, potentially allowing for the generated
documentation to display information about the usage of functions. The downside
to this is that LLVM would only be usable with languages that have a front-end
implemented for it, whereas Treetop could be used to parse any language. With
hindsight, I would have at least considered using LLVM instead of, or in
combination with, Treetop to do my parsing.

Overall I believe that my project was a modest success, despite not doing
everything I set out to do, it does go a long way to being a competent
documentation generator.
