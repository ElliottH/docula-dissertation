% The Introduction should explain the principal motivation for the project.
% Show how the work fits into the broad area of surrounding Computer Science and
% give a brief survey of previous related work. It should generally be
% unnecessary to quote at length from technical papers or textbooks. If a simple
% bibliographic reference is insufficient, consign any lengthy quotation to an
% appendix.

\chapter{Introduction}
Whilst working on several large projects, I have noticed the difference good
documentation can make to the further development or maintenance of a code base.
Documentation can save hours of tedious work finding the areas of the source
code that require work, without even considering the benefits it also brings by
better informing the developer about the inner workings of the code itself. Good
documentation informs the reader of what a particular section of code does and,
depending on the situation, what design decisions were made and how the code
itself works. When written properly, documentation should inform the reader in
enough detail that they can begin using the code without having to dig into its
internals.

The problem with writing good documentation is that it requires developer time
\& skill to maintain, unfortunately most developers have neither. Documentation
generation helps to solve this problem, by minimising the amount of work a
developer has to do to produce documentation, meaning that they are more likely
to create useful documentation. First and foremost, documentation generation
allows for developers to write documentation in the source files themselves,
where it is later extracted by the generation software; this means that the
documentation is much more convenient to write, since it can be written
alongside the code. The additional benefit from this is that the generation
software can extract the information about the section being documented, such as
its type, and include this information in the documentation itself; this reduces
the amount of work the developer has to do, whilst also making it harder for the
documentation to become inconsistent with the code.

Software, such as doxygen\cite{website:doxygen} and
Javadoc\cite{website:javadoc}, already exists to perform these tasks and goes a
long way to making the lives of developers easier. These pieces of software, and
their alternatives, are widely used, but they offer an incomplete solution to
the problems they address; they focus on helping developers establish
\emph{what} a particular thing does, or is used for, but they do not try to
document the design decisions that led to this implementation nor the
considerations that drove them.

The intention of my project was to address these shortcomings.
