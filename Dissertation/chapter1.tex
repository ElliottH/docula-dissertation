% The Introduction should explain the principal motivation for the project.
% Show how the work fits into the broad area of surrounding Computer Science and
% give a brief survey of previous related work. It should generally be
% unnecessary to quote at length from technical papers or textbooks. If a simple
% bibliographic reference is insufficient, consign any lengthy quotation to an
% appendix.

\chapter{Introduction}

\emph{\textbf{TODO:} Outline how well the project went.}

Whilst working on several large projects, I noticed that the difference between
good documentation and bad documentation (or lack thereof) can make a huge
difference to further development or maintenance of the code base; so much so,
that documentation can save hours' worth of tedious work just finding the areas
of the source code that require work, without even considering the benefits -
in both time and effort saved - it also brings by better informing a developer
or maintainer about the inner workings of the code itself. Documentation, even
in its simplest form, helps to alleviate this; something as basic as just adding
comments to the code to explain what sections do assists future developers in
getting to grips with your code.

Documentation generation goes further than this: it outlines a way in which
users should structure their comments to produce documentation, and by doing so
they can reap the benefits of documentation that is formatted for easy
consumption - e.g.~as a HTML document - and is collected together and properly
cross-referenced. This allows a user to easily jump between related sections of
the documentation, and therefore in turn jump through the code, such as a
function definition that refers to a user-defined type providing a link to the
documentation of that type.

Software that performs these tasks already exists, and goes a long way to making
the lives of developers easier, such as doxygen\cite{website:doxygen} and
Javadoc\cite{website:javadoc}. These pieces of software, and their alternatives,
are widely used, but they offer an incomplete solution to the problems they
address. Most documentation generation focuses on helping developers establish
\emph{what} a particular thing does, or is used for, but they ignore \emph{why}
it is implemented in that particular way and what considerations were made to
design it in that way.

The intention of my project was to implement a documentation generation engine,
with the plan of taking it further to tackle this problem.
