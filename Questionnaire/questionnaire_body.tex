
\hypersetup{
  linkcolor=black,
  linkbordercolor={0 0 0},
  colorlinks=false
}

\def\LayoutTextField#1#2{\makebox[6em][l]{#1}\raisebox{-.5ex}{#2}}
\def\LayoutChoiceField#1#2{\makebox[6em][l]{#1}#2}
\newdimen\longline
\longline=\textwidth\advance\longline-6em

\makeatletter
\patchcmd{\HyField@FlagsRadioButton}{\HyField@SetFlag{Ff}{Radio}}{}{}{}
\makeatother
\def\DefaultOptionsofRadio{print}

  \begin{center}
    \bf Docula: Evaluation Questionnaire
  \end{center}
  Along with this questionnaire you should have received a .gem file; to allow
  for convenient use of docula whilst testing it functionality, please execute
  the following in a shell\footnote{Depending on your Ruby setup, this may need
  to be executed as root}:
  \begin{verbatim}
    $ gem install --local docula-*.gem
  \end{verbatim}
  This will properly install docula and its dependencies onto your system. You
  should then be able to use the \verb|docula| command to generate documentation
  for a particular directory. The resulting documentation should be available in
  the \verb|output/| directory by default.

  When testing has completed, you can remove docula with:
  \begin{verbatim}
    $ gem uninstall docula
    $ gem cleanup
  \end{verbatim}
  The second command will remove all the dependencies that were installed,
  provided they are no longer required, leaving your system as it was before
  testing.

  To remove any traces of docula from a directory you used it in, simply:
  \begin{verbatim}
    $ rm -rf .docula/ output/
  \end{verbatim}

  The rest of this document contains some questions that will help me to
  evaluate the functionality of \verb|docula|, please take the time to fill them
  out; all non-multiple choice questions are optional.

  \newpage

  \begin{Form}
    \noindent
    Were you able to install the software using the instructions on the previous
    page?
    \medskip
    \begin{center}
      \ChoiceMenu[radio,name=install]{}{Yes, No}\\
    \end{center}

    \bigskip
    \noindent
    \verb|docula| can generate documentation for C source \& header files, if
    you were to run it on a directory containing these files, it should produce
    documentation.\\
    Were you able to generate successfully generate any documentation?
    \medskip
    \begin{center}
      \ChoiceMenu[radio,name=generate]{}{Yes, No}\\
    \end{center}

    \bigskip
    \noindent
    When running \verb|docula|, did any of your source files fail to be parsed?
    If so, how many?
    \medskip
    \begin{center}
      \ChoiceMenu[radio,name=generate]{}{None, A Few, A Lot, Most, All}\\
    \end{center}

    \bigskip
    \noindent
    How did the speed compare to your usual documentation generation software?
    \medskip
    \begin{adjustwidth}{-2cm}{-2cm}
    \begin{center}
      \medskip
      \ChoiceMenu[radio,name=speed]{}{Much Slower, Slower, About the Same,
      Quicker, Much Quicker}\\
      \bigskip
    \end{center}
    \end{adjustwidth}

    \bigskip
    \noindent
    \verb|docula| uses the checksums of files to determine whether documentation
    needs to be regenerated when it is rerun. If you make a change to one or
    two files in your directory, \verb|docula| should only need to re-parse
    those files.\\
    How did the speed compare to your usual documentation generation software on
    the second execution?
    \medskip
    \begin{adjustwidth}{-2cm}{-2cm}
    \begin{center}
      \medskip
      \ChoiceMenu[radio,name=speed2]{}{Much Slower, Slower, About the Same,
      Quicker, Much Quicker}\\
      \bigskip
    \end{center}
    \end{adjustwidth}

    \bigskip
    \noindent
    How does the documentation produce compare with your usual documentation
    generation software?
    \medskip
    \begin{adjustwidth}{-2cm}{-2cm}
    \begin{center}
      \medskip
      \ChoiceMenu[radio,name=doccompare]{}{Much Better, Better, About the Same,
      Worse, Much Worse}\\
    \end{center}
    \end{adjustwidth}
    \medskip
    Why?\\
    \medskip
    \TextField[name=doccomment,multiline=true,width=\longline,height=2in,
      borderwidth=0,backgroundcolor={.85 .85 .85}]{}\\

    \bigskip
    \noindent
    Was there anything in particular that you felt was missing from the
    documentation?\\
    \medskip
    \TextField[name=missingcomment,multiline=true,width=\longline,height=2in,
      borderwidth=0,backgroundcolor={.85 .85 .85}]{}\\

    \bigskip
    \noindent
    Additional Comments\\
    \medskip
    \TextField[name=additionalcomment,multiline=true,width=\longline,height=2in,
      borderwidth=0,backgroundcolor={.85 .85 .85}]{}\\
  \end{Form}

  \begin{center}
    \bigskip\hrule\bigskip
    \emph{Thank you for taking the time to complete this questionnaire.}
  \end{center}
