% Draft #1 (final?)

\vfil

\centerline{\Large Computer Science Project Proposal}
\vspace{0.4in}
\centerline{\Large How to write a dissertation in \LaTeX\ }
\vspace{0.4in}
\centerline{\large M. Richards, St John's College}
\vspace{0.3in}
\centerline{\large Originator: Dr M. Richards}
\vspace{0.3in}
\centerline{\large 14$^{th}$ October 2011}

\vfil


\noindent
{\bf Project Supervisor:} Dr M. Richards
\vspace{0.2in}

\noindent
{\bf Director of Studies:} Dr M. Richards
\vspace{0.2in}
\noindent
 
\noindent
{\bf Project Overseers:} Dr~F.~H.~King  \& Dr~A.~W.~Moore


% Main document

\section*{Introduction, The Problem To Be Addressed}


Many students write their CST dissertations in \LaTeX\ and
spend a fair amount of time learning just how to do that. The purpose of 
this project is to write a demonstration dissertation that explains in
detail how it done.  

This core proposal document will be augmented by a separately-printed
cover sheet at the front and a resource form at the end.  Additional
sheets for risk assessment and human resources may also need to be included.

This document will repeat much of the material that is summarised on the additional sheets.

\section*{Starting Point}

{\em Describe existing state of the art, previous work in this area, libraries and databases to be used.
Describe the state of any existing codebase that is to be built on.  }

I am already able to write prose using the English language. I have an online dictionary. etc..

\section*{Resources Required}

{\em A note of the resources required and confirmation of access.}

For this project I shall mainly use my own quad-core computer that runs Fedora Linux. Backup
will be to github and/or to an SVN repository on an external hard disk that is dumped to writable CD/DVD media.
I have another similar computer to hand should my main machine suddenly fail.
I require no other special resources.

\section*{Work to be done}

{\em Describe the technical work.}

The project breaks down into the following sub-projects:

\begin{enumerate}

\item The construction of a skeleton dissertation with the required 
structure. This involves writing the Makefile and makeing dummy files
for the title page, the proforma, chapters 1 to 5, the appendices and
the proposal.

\item Filling in the details required in the cover page and proforma.

\item Writing the contents of chapters 1 to 5, including examples
of common \LaTeX\ constructs.

\item Adding a example of how to use floating figures and encapsulated
postscript diagrams.

\end{enumerate}

\section*{Success Criterion for the Main Result}


The project will be a success if I have a completed dissertation with the correct chapter
titles and I have achieved my other success criterion, which is to blah ...



\section*{Possible Extensions}

{\em Potential further envisaged evaluation metrics or extensions.}

If I achieve my main result early I shall try the following alternative experiment or method of evaluation ...


\section*{Timetable: Workplan and Milestones to be achieved.}


{\em Perhaps list ten or so  two-week work-packages.}

Planned starting date is 16/10/2011.

\begin{enumerate}

\item {\bf Michaelmas weeks 2-4} Learn to use X. Read book Y. Read papers Z.

\item {\bf Michaelmas weeks 5-6} Do preliminary test of Q.

\item {\bf Michaelmas weeks 7-8} Start implementation of main task A.

\item {\bf Michaelmas vacation} Finish A and start main task B.

\item {\bf Lent weeks 0-2} Write progress report. Generate corpus of test examples. Finish task B.  

\item {\bf Lent weeks 3-5} Run main experiments and achieve working project.

\item {\bf Lent weeks 6-8} Second main deliverable here.

\item {\bf Easter vacation:} Extensions and writing dissertation main chapters.

\item {\bf Easter term 0-2:}  Further evaluation and complete dissertation.

\item {\bf Easter term 3:} Proof reading and then an early submission so as to concentrate on examination revision.

\end{enumerate}


 

