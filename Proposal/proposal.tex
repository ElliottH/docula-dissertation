\section{Introduction, The Problem To Be Addressed}

In the development and maintenance of software, especially those
containing large amounts of code, documentation frequently becomes a
problem. Often documentation is written separately from the code
itself, and as such the two diverge over time, leaving maintainers
worse of than they'd never checked both in the first place. It is the
aim of this project to tackle this problem and to facilitate not only
the writing of correct and thorough documentation, but also the easy
maintenance of both code and documentation; this is to be achieved by
defining a syntax to interleave code with documentation, so that both
are kept together and so are more easily kept in synchronisation with
one another.

It is important to distinguish between the two types of documentation
that occur in software development\footnote{The naming of these is
purely of my own invention, and as far as I'm aware is not part of any
established convention; I will, however, be using these names
throughout.} Together, these two types of documentation should
describe a piece of software thoroughly:

\begin{description}
\item[Interface documentation] essentially defines a contract with the
  outside world as to what this function or class does, what it
  expects as input to give a valid output, what may (or may not)
  happen when given invalid inputs, and so on.

  It is important that \emph{interface documentation} be well-defined,
  as it is what informs another developer now or in the future as to
  what the function is \emph{supposed} to do.

\item[Implementation documentation] is essentially the comments that
  developers put (or rather, should put) into their code that explains
  \emph{how} and \emph{why} they are doing things the way they're
  doing them. These comments should be written with the expectation
  that someone else in the future may actually be using them to track
  down the relevant piece of code they looking to fix or update, as
  such they should be written as proper sentences and written in such
  a way that the comments can be combined into a full document that
  talks its way through the function.

  The \emph{implementation documentation} does not need to be as
  strictly defined as the \emph{interface documentation}, as it is
  more a process of describing the choices made in writing the
  function, rather than its intended use.
\end{description}

\section{Starting Point}

Several other documentation engines exist, such as Doxygen \& Javadoc,
however none of them seem to account for the two distinct types of
documentation that exist within a code base. I have used both Doxygen
\& Javadoc to document projects in the past, and so I am familiar with
the style in which the comments used for the documentation are
written.

Since that there is already wide use of these, particularly Doxygen,
I will be designing my syntax to be as compatible with theirs as
possible. This will allow users to easily switch between the two and
allow for increased uptake of the engine.

\section{Resources Required}

I shall be primarily be using my own computer for the development of
this project, which is currently a laptop running Mac OS X; I will
also be using the PWF machines, to ensure that the software produced
runs across a variety of UNIX-like operating systems. The project
shall be backed up to a git repository on BitBucket, and also to the
PWF.

I require no other special resources.

\section{Work to be done}

{\em Describe the technical work.}

The project breaks down into the following sub-projects:

\begin{enumerate}

\item The construction of a skeleton dissertation with the required 
structure. This involves writing the Makefile and makeing dummy files
for the title page, the proforma, chapters 1 to 5, the appendices and
the proposal.

\item Filling in the details required in the cover page and proforma.

\item Writing the contents of chapters 1 to 5, including examples
of common \LaTeX\ constructs.

\item Adding a example of how to use floating figures and encapsulated
postscript diagrams.

\end{enumerate}

\section{Success Criterion for the Main Result}


The project will be a success if I have a completed dissertation with the correct chapter
titles and I have achieved my other success criterion, which is to blah ...



\section{Possible Extensions}

{\em Potential further envisaged evaluation metrics or extensions.}

If I achieve my main result early I shall try the following alternative experiment or method of evaluation ...


\section{Timetable: Workplan and Milestones to be achieved.}


{\em Perhaps list ten or so  two-week work-packages.}

Planned starting date is 16/10/2011.

\begin{enumerate}

\item {\bf Michaelmas weeks 2-4} Learn to use X. Read book Y. Read papers Z.

\item {\bf Michaelmas weeks 5-6} Do preliminary test of Q.

\item {\bf Michaelmas weeks 7-8} Start implementation of main task A.

\item {\bf Michaelmas vacation} Finish A and start main task B.

\item {\bf Lent weeks 0-2} Write progress report. Generate corpus of test examples. Finish task B.  

\item {\bf Lent weeks 3-5} Run main experiments and achieve working project.

\item {\bf Lent weeks 6-8} Second main deliverable here.

\item {\bf Easter vacation:} Extensions and writing dissertation main chapters.

\item {\bf Easter term 0-2:}  Further evaluation and complete dissertation.

\item {\bf Easter term 3:} Proof reading and then an early submission so as to concentrate on examination revision.

\end{enumerate}


 

